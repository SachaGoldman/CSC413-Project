\documentclass{article}

\PassOptionsToPackage{numbers, compress}{natbib}

\usepackage[final]{proposal_neurips_2021}

\bibliographystyle{abbrvnat}

\usepackage[utf8]{inputenc}
\usepackage[T1]{fontenc}
\usepackage{hyperref}
\usepackage{url}
\usepackage{booktabs}
\usepackage{amsfonts}
\usepackage{nicefrac}
\usepackage{microtype}
\usepackage{xcolor}

\title{Proposal: DINO Methods for Style Transfer}

\author{
  Sacha Goldman \\
  Department of Computer Science\\ 
  Department of Mathematics\\
  University of Toronto\\
  Toronto, ON M5S 1A1 \\
  \texttt{sacha.goldman@mail.utoronto.ca} \\
  \And
  Yuchong Zhang \\
  Department of Computer Science \\
  Department of Mathematics \\
  University of Toronoto \\
  Toronto, ON, M5S 1A1 \\
  \texttt{yuchongz.zhang@mail.utoronto.ca} \\
  \And
  Shirley Wang \\
  Department of Computer Science \\
  University of Toronto \\
  Toronto, ON M5S 1A1 \\
  \texttt{shirlsyuemeng.wang@mail.utoronto.ca} \\
}

\begin{document}

\maketitle

\begin{abstract}
We combine DINO \cite{DINO}, a self-supervised learning method that captures feature representations of input images, with StyTr2 \cite{ImageStyleTransformer}, a style transfer transformer that renders images with artistic style, and explore the nature and robustness of the resulting model.
\end{abstract}

\section{Introduction}
Style transfer is the problem of changing an image's artistic style while maintaining its ``content'', namely the object it depicts. Conventional models use a CNN to extract features from an image, take in another image indicating the desired artistic style, and attempt to apply that style to the features of the original input image. However, it has been shown that due to the locality resulting from shared weights, CNN based models struggle to capture global information of input images and suffer from feature losses \cite{ImageStyleTransformer}. To resolve these issues, models that use transformers instead of CNNs for feature and style extraction have been proposed and show better results \cite{ImageStyleTransformer}. The model StyTr2 proposed in \cite{ImageStyleTransformer} takes in a content image and a style image, uses two transformer encoders to capture the content and style, then renders the content with that style using a thrid transformer decoder. In our project, we will substitute the DINO method and models from \cite{DINO} for the encoder components and explore the effects on the task of style transfer. The DINO method is good at extracting content even if the input images are distorted or partial, so in particular, we will implement the models and conduct experiments to explore the following questions:
\begin{enumerate}
  \item By using DINO to extract content, can we obtain a style transfer that is more robust to distortion of the content image?
  \item Does DINO's ability to capture an image's global property from local pieces give it more robustness as a style encoder?
  \item Are the results of this new style transfer method better than previous methods?
\end{enumerate}


\section{Related Works}

\cite{DINO} came up with a new self-supervised learning method, named DINO (for Knowledge Distillation with No Labels) that aims to create better feature representations from its inputs by adopting a teacher-student paradigm. The student gets a difficult version (lots of augmentation) of the image while the teacher gets an easy version (minimal augmentation), and then the models are trained to aim for the same feature representation through cross entropy loss. Their experimental results found that the vision transformer's class token contained explicit information about the segmentation of objects in images, while also performing as very accurate classifiers. In this project, we aim to apply the better feature representation that the DINO method yields (over supervised vision transformers), to the domain of image style transfer.

Self-supervised learning is a common technique used in style transfer, due to the obvious lack of target images available for a task of this nature. It becomes necessary to find ways for the model to supervise its own outputs. Usual methods \cite{CNNStyleTransfer} involve checking the outputs of VGG for consistency in style and context. \cite{ImageStyleTransformer} proposes using vision transformers for this task, using two transformer encoders (one for content, one for style), and one transformer decoder. Transformers have been a hot rising topic in computer vision, and they claim that their method of using transformers for style transfer results in a decrease in content leak, as well as acheives unbiased stylization.

It becomes integral for there to be good feature representations of the content and style to construct a good style-transferred image. We believe that making use of pretrained DINO ViT models, as well as using the DINO method itself during training, can help achieve this.


\section{Method}

We have four different ideas for bringing the success of the DINO method to the style transfer task.

\subsection{Frozen DINO Content Encoder}

We initialize the content encoder with the DINO pretrained vision transformer and freeze its weights. Then we train the style encoder and decoder using the regular style transfer training paradigm from \cite{ImageStyleTransformer}. The idea being that the DINO model's robust feature vectors may act as powerful abstractions of the content of the images, that can be used for creating a new stylized image.

\subsection{Initial Pretrained DINO Content Encoder}

We initialize the content encoder with the DINO pretrained vision transformer, but we run the style transfer training paradigm on both encoders and the decoder. We expect this to converge to a different optima because transformers have many local optima. We hope that the robust abstractions from the DINO training process lead to better style transfer after training.

\subsection{DINO Training for Encoders}

For the purpose of DINO training we add a teacher content encoder and a teacher style encoder. Now, as in the DINO method, the content and style transformers are given distorted versions of the images (possibly only cropped and flipped to avoid abstracting the incorrect style characteristics), and the teachers are given the unmodified images. We then add together the loss from the standard style transfer and the cross entropy loss between the teacher encodings and the student encodings. This forces the students to learn distillations of the original images. We expect this to work better for the style encoder, because style is a global property of an image, so we may only apply this method for the style encoder.

\subsection{Modified DINO Training for Encoders}

This method is similar to the previous, except that instead of taking the cross entropy between the encodings, we take apply the decoder transformer to the encodings from both the teacher and student encoders, which would yield a teacher image and a student image. We then use the distance between the teacher image and the student image in our loss. The precise distance metric used here is to be determined through experimentation. This will allow the encoders to distill the important pieces of the stylized image. This method may suffer from being very expensive to train. We also may only apply this to the style encoder.

\medskip

% Our current sources
\nocite{*}

\bibliography{bib}

\end{document}