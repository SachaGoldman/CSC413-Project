\documentclass{article}

\PassOptionsToPackage{numbers, compress}{natbib}

\usepackage[final]{proposal_neurips_2021}

\bibliographystyle{abbrvnat}

\usepackage[utf8]{inputenc}
\usepackage[T1]{fontenc}
\usepackage{hyperref}
\usepackage{url}
\usepackage{booktabs}
\usepackage{amsfonts}
\usepackage{nicefrac}
\usepackage{microtype}
\usepackage{xcolor}

\title{Proposal: DINO Methods for Style Transfer}

\author{
  Sacha Goldman \\
  Department of Computer Science\\ 
  Department of Mathematics\\
  University of Toronto\\
  Toronto, ON M5S 1A1 \\
  \texttt{sacha.goldman@mail.utoronto.ca} \\
  % \And
  % Coauthor \\
  % Affiliation \\
  % Address \\
  % \texttt{email} \\
  % \And
  % Coauthor \\
  % Affiliation \\
  % Address \\
  % \texttt{email} \\
}

\begin{document}

\maketitle

\begin{abstract}

\end{abstract}

\section{Introduction}
Style transfer is the problem of changing an image's artistic style while maintaining its ``content'', namely the object it depicts. Conventional models use CNN to extract features from an image, take in another image indicating the desired artistic style, and attempt to apply that style to the features of the original input image. However, it has been shown that due to the locality resulting from shared weights, CNN struggles to capture global information of input images and suffers from feature losses \cite{ImageStyleTransformer}. To resolve these issues, models that use Transformers instead of CNN for feature and style extraction have been proposed and shown improved results \cite{ImageStyleTransformer}. The authors of \cite{ImageStyleTransformer} use 


\section{Related Works}



\section{Method}

We have four different idea for brining the success of the DINO method to the domain of style transfer.

\subsection{DINO Content Encoder} 

\subsection{Pre-trained DINO Content Encoder}

\subsection{DINO Training for Encoders}

\subsection{Modified DINO Training for Encoders}

\medskip

% Our current sources
\nocite{*}

\bibliography{bib}

\end{document}